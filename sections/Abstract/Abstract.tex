Dette skal være et sammendrag av hele rapporten. Her skal det altsa bade være noen ord om hva som eventuelt er gjort tidligere innefor fagfeltet, hva man selv har gjort for a bringe feltet videre og ikke minst de viktigste resultatene man har oppnadd. Hovedfokus vil naturlig nok ligge pa de to siste punktene. Det vil ofte være sammendraget som avgjør hvor interessant noen finner det a begynne a lese rapporten din, og skriver du et godt sammendrag, hvor du kan pirre sensors interesse, har du kommet langt pa vei til en god karakter.\\

Denne rapporten omhandler Euler-Bernoulli bjelken (The Euler-Bernoulli beam). Dette er en fundamental modell som sier hvordan forskjellige materialer bøyer seg under påvirkning av diverse krefter. BLABLABLA MER OM DETTE ?!?! HVA SKRIVER MAN. 
Euler-Bernoulli modellen er en meget sentral modell innenfor dette området, og det er skrevet mange artikler rundt dette.

I rapporten går vi først igjennom grunnleggende teori som utgjør grunnlaget for utregningene gjort for å regne ut den vertikale forskyvningen i hvert punkt langs bjelken. På grunn av at oppgaven vi har fått er en faktisk oppgave som står i læreboken [KILDE HER KANSKJE?! :D ], går vi i teoridelen igjennom de fleste formlene boken oppgir, med noen få unntak som spesifikt ikke skulle beskrives. Til slutt i teoridelen følger det 2 bevis. Det ene beviset beviser likning 2.28, som er et uttrykk på den fjerdederiverte til funksjonen y i Euler-Bernoulli likningen. Av grunner beskrevet i teoridelen er ikke denne likningen brukbar langs hele planken, og derfor følger et nytt bevis på den fjerdederiverte i punktet $x_1$ etter dette. \\

Etter teoridelen løser vi en rekke oppgaver. Oppgavene vi løser er oppgavene som står til slutt i Reality Check 2 (HVORDAN SIER MAN DETTE?!), og går ut på forskjellige utregninger knyttet til et stupebrett, som enkelt kan relateres til Euler-Bernoulli-likningen. Oppgavene går fra å lage diverse MatLab-programmer som regner ut den vertikale forskyvningen langs planken, på forskjellige måter, til både det å finne hvor store feil forskjellige funksjoner har. Vi får også oppgitt en løsning på funksjonen, som vi skal vise at tilfredsstiller Euler-Bernoulli funksjonen, gitt en viss påført kraft. Til slutt regner vi også den vertikale forskyvningen når vi påfører vekt på stupebrettet. 

KILDER: 

