\subsection{Exercise 5.1.21} 
\label{sec:exercise_5_1_21}
Følgende kommer et bevis på at den fjerdederiverte (likning 2.28) kan approksimeres ved: 

\begin{align}
    f^{iv}(x)=\frac{f(x-2h)-4f(x-h)+6f(x)-4f(x+h)+f(x+2h)}{h^4} \label{eq21:proove}
\end{align}

Først setter vi opp taylor-rekker for punktene $x+2h$, $x-2h$, $x+h$, $x-h$: 
\begin{align}
    f(x+2h)&=f(x)+2hf'(x)+2h^2f''(x)+\frac{(2h)^3}{3!}f'''(x)+\frac{(2h)^4}{4!}f^{(4)}(x)+\frac{(2h)^5}{5!}f^{(5)}(x)+h^6\nonumber\\ 
    f(x-2h)&=f(x)-2hf'(x)+2h^2f''(x)-\frac{(2h)^3}{3!}f'''(x)+\frac{(2h)^4}{4!}f^{(4)}(x)-\frac{(2h)^5}{5!}f^{(5)}(x)+h^6\nonumber\\
    f(x+h)&=f(x)+hf'(x)+\frac{h^2f''(x)}{2}+\frac{h^3}{3!}f'''(x)+\frac{h^4}{4!}f^{(4)}(x)+\frac{h^5}{5!}f^{(5)}(x)+h^6\nonumber\\
    f(x-h)&=f(x)-hf'(x)+\frac{h^2f''(x)}{2}-\frac{h^3}{3!}f'''(x)+\frac{h^4}{4!}f^{(4)}(x)-\frac{h^5}{5!}f^{(5)}(x)+h^6\label{Theory:taylorrekker}
\end{align}

Deretter legger vi først sammen $f(x+2h)$ og $f(x-2h)$: 

\begin{multline}
    \\
    f(x+2h)+f(x-2h)= \\
    f(x)+2hf'(x)+2h^2f''(x)+\frac{(2h)^3}{3!}f^3(x)+\frac{(2h)^4}{4!}f^{4}(x)+\frac{(2h)^5}{5!}f^{(5)}(x)+h^6 \\
    + \\
  	f(x)-2hf'(x)+2h^2f''(x)-\frac{(2h)^3}{3!}f^3(x)+\frac{(2h)^4}{4!}f^{4}(x)-\frac{(2h)^5}{5!}f^{(5)}(x)+h^6\nonumber \\
  	=\\
  	2f(x)+4h^2f''(x)+\frac{4}{3}h^4f^{(4)}(x)+h^6 \nonumber\\
\end{multline}

Deretter legger vi sammen $f(x+h)$ og $f(x-h)$: 
\begin{multline}
	\\
  f(x)+hf'(x)+\frac{h^2f''(x)}{2}+\frac{h^3}{3!}f'''(x)+\frac{h^4}{4!}f^{(4)}(x)+\frac{h^5}{5!}f^{(5)}(x)+h^6\\
	+\\
	f(x)-hf'(x)+\frac{h^2f''(x)}{2}-\frac{h^3}{3!}f'''(x)+\frac{h^4}{4!}f^{(4)}(x)-\frac{h^5}{5!}f^{(5)}(x)+h^6\\
	=\\
	2f(x)+h^2f''(x)+\frac{h^4f^{(4)}(x)}{12} + h^6 \nonumber \\
\end{multline}

Til slutt legger vi sammen $f(x+h)+f(x-h)$ og $\frac{f(x+2h)+f(x-2h)}{-4}$: 
\begin{multline}
	\\
	2f(x)+h^2f''(x)+\frac{h^4f^{(4)}(x)}{12} + h^6 \\
	+\\
	\frac{2f(x)+4h^2f''(x)+\frac{4}{3}h^4f^{(4)}(x)+h^6 }{-4}\\
	=\\
	\frac{3}{2}f(x)-\frac{3}{12}h^4f^{(4)}(x)+h^6\\
\end{multline}

Deretter løser vi likningen med hensyn på $f^{(4)}(x)$: 
\begin{align}
    f(x+h)+f(x-h)-\frac{f(x+2h)}{4}-\frac{f(x-2h)}{4}=\frac{3}{2}f(x)-\frac{3}{12}h^4f^{(4)}(x)+h^6\nonumber
\end{align}
\begin{align}
     -3h^4f^{(4)}(x)&=12f(x+h)+12f(x-h)-3f(x+2h)-3f(x-2h)-18f(x) +h^6 \nonumber \\
     \nonumber \\
	f^{(4)}(x)&=\frac{-4f(x-h)+f(x-2h)+6f(x)+f(x+2h)-4f(x+h)}{h^4} \label{eq21a:prooved}
\end{align}
 

 

% Her kommer det andre beviset: 
% \begin{align}
% f(x+2h)+f(x-2h)+f(x+h)+f(x-h)=f4(x)+5h^2f''(x)+\frac{17}{12}h^{(4)}(x)+h^6 \nonumber
% \end{align}

% Vi setter $f^{(4)}$ alene, og får: 

% \begin{align}
% 	f^{(4)}=\frac{12f(x+2h)}{17h^4}+\frac{12f(x-2h)}{17h^4}+\frac{12f(x+h)}{17h^4}+\frac{12f(x-h)}{17h^4}-\frac{48f(x)}{17h^4}-\frac{60f''(x)}{17h^2}+\frac{h^6}{h^4}
% \end{align}

% Vi ser at vi trenger $f''(x)$. HER SKAL VI FINNE DEN; HVORDAN GJØR VI DET?!??!?!?! 
% Setter vi dette inn i uttrykket vi har for $f^{(4)}(x)$ får vi: 

% \begin{multline}
%     f^{(4)}(x)=\frac{12f(x+2h)}{17h^4}+\frac{12f(x-2h)}{17h^4}+\frac{12f(x+h)}{17h^4}+\frac{12f(x-h)}{17h^4}-\frac{48f(x)}{17h^4}-\\ \frac{60f}{17h^2}\cdot (\frac{-f(x+2h)+16f(x+h)-30f(x)+16f(x-h)-f(x-h)}{12h^2})+\frac{h^6}{h^4} \nonumber
% \end{multline}

% \begin{multline}
%     f^{(4)}(x)=\frac{12f(x+2h)}{h^4}+\frac{12f(x-2h)}{h^4}+\frac{12f(x+h)}{h^4}+\frac{12f(x-h)}{h^4}-\frac{48f(x)}{h^4} \\
%     +\frac{5f(x+2h)}{17h^4}-\frac{80f(x+h)}{17h^4}+\frac{150f(x)}{17h^4}-\frac{80f(x-h)}{17h^4}+\frac{5f(x-2)}{17h^4} + \frac{h^6}{h^4} \nonumber
% \end{multline}

% \begin{align}
%     f^{(4)}(x)&=\frac{f(x+2h-4f(x+h)+6f(x)-4f(x-h)+f(x-2h)}{h^4} + h^2
% \end{align}