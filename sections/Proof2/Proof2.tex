% !TEX encoding = UTF-8 Unicode
 
\section*{Exercise 5.1.22a}
\label{sec:oppgave22a}

Bevis at hvis $f(x) = f'(x) = 0$, så vil 
\begin{multline}
{f^{iv}}(x + h) - \frac{{16f(x + h) - 9f(x + 2h) + \frac{8}{3}f(x + 3h) - \frac{1}{4}f(x + 4h)}}{{{h^4}}} = O({h^2})  
\label{eq22a:oppg} \\
\end{multline}

Vi får oppgitt i oppgaveteksten at hvis $f(x) = f'(x) = 0$, er
\begin{align}
f(x-h)-10f(x+h)+5f(x+2h)-\frac{5}{3}f(x+3h)+\frac{1}{4}f(x+4h)=O(h^6) 
\label{eq22a:hint} \\ \nonumber
\end{align}

I \ref{Theory:taylorrekker}, fant vi taylorrekkene for punktene $x-h$, $x+h$, $x+2h$. Vi finner her taylorrekkene for punktene $x+3h$ og $x+4h$: 

\begin{align}
f(x+3h) &= f(x) + 3hf'(x) + \frac{{9{h^2}}}{2}f''(x) + 3{h^3}f'''(x) + \frac{{27{h^4}}}{8}{f^{iv}}(x) + \frac{{81{h^5}}}{40}{f^v}(x) + {h^6} \nonumber \\ 
f(x+4h) &= f(x) + 4hf'(x) + 8{h^2}f''(x) + \frac{{32{h^3}}}{3}f'''(x) + \frac{{32{h^4}}}{3}{f^{iv}}(x) + \frac{{128{h^5}}}{{15}}{f^v}(x) + {h^6}\nonumber \\ \nonumber
\end{align}


Setter så inn koeffisientene vi ble gitt i hintet foran taylor-rekkene. 
\begin{align}
 f(x-h)&=f(x)-hf'(x)+\frac{h^2f''(x)}{2}-\frac{h^3}{6}f'''(x)+\frac{h^4}{24}f^{(4)}(x)-\frac{h^5}{120}f^{(5)}(x)+O(h^6)\nonumber \\
10f(x + h) &= 10f(x) + 10hf'(x) + 5{h^2}f''(x) + \frac{{5{h^3}}}{3}f'''(x) + \frac{{5{h^4}}}{{12}}{f^{(4)}}(x) + \frac{{{h^5}}}{{12}}{f^{(5)}}(x) + O(h^6) \nonumber\\
5f(x + 2h) &= 5f(x) + 10hf'(x) + 10{h^2}f''(x) + \frac{{20{h^3}}}{3}f'''(x) + \frac{{10{h^4}}}{3}{f^{(4)}}(x) + \frac{{4{h^5}}}{3}{f^{(5)}}(x) + O(h^6) \nonumber \\
\frac{5}{3}f(x + 3h) &= \frac{5}{3}f(x) + 5hf'(x) + \frac{{15{h^2}}}{2}f''(x) + \frac{15h^3}{2}f'''(x) + \frac{{45{h^4}}}{8}{f^{iv}}(x) + \frac{{27{h^5}}}{8}{f^v}(x) + O(h^6) \nonumber \\ 
\frac{1}{4}f(x + 4h) &= \frac{1}{4}f(x) + hf'(x) + 2{h^2}f''(x) + \frac{{8{h^3}}}{3}f'''(x) + \frac{{8{h^4}}}{3}{f^{iv}}(x) + \frac{{32{h^5}}}{{15}}{f^v}(x) + O(h^6) \nonumber 
\end{align}
\newpage


Siden $f(x) = f'(x) = 0$, kan uttrykk med $f(x)$ og $f'(x)$, i Taylorrekkene fjernes. Vi setter inn disse Taylorrekkene i hintet gitt i oppgaven \ref{eq22a:hint}: 
\begin{multline}
\\ \frac{h^2}{2}f''(x)-\frac{h^3}{6}f'''(x)+\frac{h^4}{24}f^{(4)}(x)-\frac{h^5}{120}f^{(5)}(x)+{O_1}(h^6) \\
-\\
(5{h^2}f''(x) + \frac{{5{h^3}}}{3}f'''(x) + \frac{{5{h^4}}}{{12}}{f^{(4)}}(x) + \frac{{{h^5}}}{{12}}{f^{(5)}}(x) + {O_2}(h^6)) \\
+ \\
10{h^2}f''(x) + \frac{{20{h^3}}}{3}f'''(x) + \frac{{10{h^4}}}{3}{f^{(4)}}(x) + \frac{{4{h^5}}}{3}{f^{(5)}}(x) + {O_3}(h^6) \\
- \\
(\frac{{15{h^2}}}{2}f''(x) + 5{h^3}f'''(x) + \frac{{45{h^4}}}{8}{f^{iv}}(x) + \frac{{15{h^5}}}{4}{f^v}(x) + {O_4}(h^6)) \\
+ \\
2{h^2}f''(x) + \frac{{8{h^3}}}{3}f'''(x) + \frac{{8{h^4}}}{3}{f^{iv}}(x) + \frac{{32{h^5}}}{{15}}{f^v}(x) + {O_5}(h^6) \\
= \\
0f''(x)+0f'''(x)+0f^{iv}(x)+0f^{v}(x)+O(h^6) \\
= \\
O(h^6) \\
\end{multline}


Har da vist at hintet gitt i oppgaven stemmer. \\ 
Vi endrer så dette uttrykket til å være lik $O(h^2)$, likt som i likning \ref{eq22a:oppg}. 
\begin{align}
\frac{f(x-h)-10f(x+h)+5f(x+2h)-\frac{5}{3}f(x+3h)+\frac{1}{4}f(x+4h)}{h^4}=O(h^2) 
\label{eq22a:deleH} \\ \nonumber
\end{align}


Vi endrer uttrykket vi kom fram til i \ref{eq21:proove} fra ${f^{iv}}(x)$ til ${f^{iv}}(x + h)$ og får: 
\begin{align}
{f^{iv}}(x + h) = \frac{{f(x - h) - 4f(x) + 6f(x + h) - 4f(x + 2h) + f(x + 3h)}}{{{h^4}}} + O({h^2}) \label{eq22a:plussH} 
\end{align}


Utrykket funnet i \ref{eq22a:deleH} settes inn i uttrykket vi fant i \ref{eq22a:plussH}:
\begin{multline}
\\{f^{iv}}(x + h) - O(h^2) \\
= \\
\frac{f(x-h)-4f(x)+6f(x+h)-4f(x+2h)+f(x+3h)}{h^4}+O(h^2) \\
- \\
\frac{f(x-h)-10f(x+h)+5f(x+2h)-\frac{5}{3}f(x+3h)+\frac{1}{4}f(x+4h)}{h^4} \\
= \\
\frac{-4f(x)+16f(x+h)-9f(x+2h)+\frac{8}{3}f(x+3h)+\frac{1}{4}f(x+4h)}{h^4}+O(h^2) \\ \nonumber \\ \nonumber
\end{multline}

% DRIVER HER: 

Siden $f(x) = f'(x) = 0$, kan deler fra uttrykket fjernes. Vi får
\begin{multline}
\frac{16f(x+h)-9f(x+2h)+\frac{8}{3}f(x+3h)+\frac{1}{4}f(x+4h)}{h^4}+O(h^2) \label{eq22a:nesten} \\
\end{multline}

Vi ser at det uttrykket er veldig likt uttrykket vi har i \ref{eq22a:oppg}. Vi endrer på uttrykket ${f^{iv}}(x + h) - ({f^{iv}}(x + h) - O(h^2))$ som ble funnet it \ref{eq22a:nesten}. 
\begin{multline}
\\ {f^{iv}}(x + h) - ({f^{iv}}(x + h) - O(h^2))  \\
= \\
{f^{iv}}(x + h) - (\frac{16f(x+h)-9f(x+2h)+\frac{8}{3}f(x+3h)+\frac{1}{4}f(x+4h)}{h^4}+O(h^2)) \\ \nonumber
\end{multline}


Vi skriver til slutt om uttrykket:
\begin{multline}
\\ {f^{iv}}(x + h) - \frac{16f(x+h)-9f(x+2h)+\frac{8}{3}f(x+3h)+\frac{1}{4}f(x+4h)}{h^4} = O(h^2) \label{eq22a:ferdig}
\end{multline}
Og dermed er oppgaven bevist. 




