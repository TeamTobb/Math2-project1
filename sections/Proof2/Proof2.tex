% !TEX encoding = UTF-8 Unicode


\section*{Exercise 5.1.21a}
\label{sec:oppgave21a}

Bevis at hvis $f(x) = f'(x) = 0$, så vil 
\[{f^{iv}}(x + h) - \frac{{16f(x + h) - 9f(x + 2h) + \frac{8}{3}f(x + 3h) - \frac{1}{4}f(x + 4h)}}{{{h^4}}} = O({h^2})\].
Vi endrer uttrykket vi kom fram til i oppgave 5.1.21 fra ${f^{iv}}(x)$ til ${f^{iv}}(x + h)$: 
\begin{align}
{f^{iv}}(x + h) = \frac{{f(x - h) - 4f(x) + 6f(x + h) - 4f(x + 2h) + fx + 3h}}{{{h^4}}} + O({h^2}) \nonumber \\
 \nonumber \\ \nonumber
\end{align}

NYTT HER: 

\section*{Exercise 5.1.21a}
\label{sec:oppgave21a}



Bevis at hvis $f(x) = f'(x) = 0$, så vil 
\begin{align}
{f^{iv}}(x + h) - \frac{{16f(x + h) - 9f(x + 2h) + \frac{8}{3}f(x + 3h) - \frac{1}{4}f(x + 4h)}}{{{h^4}}} = O({h^2})
\end{align}


Vi får oppgitt i oppgaveteksten at hvis $f(x) = f'(x) = 0$, er
\begin{align}
f(x-h)-10f(x+h)+5f(x+2h)-\frac{5}{3}f(x+3h)+\frac{1}{4}f(x+4h)=O(h^6)) \nonumber \\
\frac{f(x-h)-10f(x+h)+5f(x+2h)-\frac{5}{3}f(x+3h)+\frac{1}{4}f(x+4h)}{h^4}=O(h^2) \\
\end{align}


Vi endrer uttrykket vi kom fram til i oppgave 5.1.21 fra ${f^{iv}}(x)$ til ${f^{iv}}(x + h)$: 
\begin{align}
{f^{iv}}(x + h) = \frac{{f(x - h) - 4f(x) + 6f(x + h) - 4f(x + 2h) + fx + 3h}}{{{h^4}}} + O({h^2}) 
\end{align}

Vi skrives så om utrykket ${f^{iv}}(x + h) - O(h^2) = g(x)$ og får\nonumber \\ ${f^{iv}}(x + h) - g(x) = O(h^2)$:
\begin{multline}
\\{f^{iv}}(x + h) - g(x) = O(h^2) \\ \\
\frac{f(x-h)-4f(x)+6f(x+h)-4f(x+2h)+f(x+3h)}{4}+O(h^2) \\
- \\
\frac{f(x-h)-10f(x+h)+5f(x+2h)-\frac{5}{3}f(x+3h)+\frac{1}{4}f(x+4h)}{h^4} \\
= \\
\frac{-4f(x)+16f(x+h)-9f(x+2h)+\frac{8}{3}f(x+3h)+\frac{1}{4}f(x+4h)}{h^4}+O(h^2) \\
\end{multline}
Siden $f(x) = f'(x) = 0$, kan deler fra utrykket fjernes. Vi får
\begin{multline}
\frac{16f(x+h)-9f(x+2h)+\frac{8}{3}f(x+3h)+\frac{1}{4}f(x+4h)}{h^4}+O(h^2) \\
\end{multline}
